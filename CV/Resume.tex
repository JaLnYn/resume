
%-------------------------
% Resume in Latex
% Author : Jake Gutierrez
% Based off of: https://github.com/sb2nov/resume
% License : MIT
%------------------------

\documentclass[letterpaper,11pt]{article}

\usepackage{latexsym}
\usepackage[empty]{fullpage}
\usepackage{titlesec}
\usepackage{marvosym}
\usepackage[usenames,dvipsnames]{color}
\usepackage{verbatim}
\usepackage{enumitem}
\usepackage[hidelinks]{hyperref}
\usepackage{fancyhdr}
\usepackage[english]{babel}
\usepackage{tabularx}
\usepackage{hyperref}
\input{glyphtounicode}


%----------FONT OPTIONS----------
% sans-serif
% \usepackage[sfdefault]{FiraSans}
% \usepackage[sfdefault]{roboto}
% \usepackage[sfdefault]{noto-sans}
% \usepackage[default]{sourcesanspro}

% serif
% \usepackage{CormorantGaramond}
% \usepackage{charter}


\pagestyle{fancy}
\fancyhf{} % clear all header and footer fields
\fancyfoot{}
\renewcommand{\headrulewidth}{0pt}
\renewcommand{\footrulewidth}{0pt}

% Adjust margins
\addtolength{\oddsidemargin}{-0.5in}
\addtolength{\evensidemargin}{-0.5in}
\addtolength{\textwidth}{1in}
\addtolength{\topmargin}{-.5in}
\addtolength{\textheight}{1.0in}

\urlstyle{same}

\raggedbottom
\raggedright
\setlength{\tabcolsep}{0in}

% Sections formatting
\titleformat{\section}{
  \vspace{-4pt}\scshape\raggedright\large
}{}{0em}{}[\color{black}\titlerule \vspace{-5pt}]

% Ensure that generate pdf is machine readable/ATS parsable
\pdfgentounicode=1

%-------------------------
% Custom commands
\newcommand{\resumeItem}[1]{
  \item\small{
    {#1 \vspace{-2pt}}
  }
}

\newcommand{\resumeSubheading}[4]{
  \vspace{-2pt}\item
    \begin{tabular*}{0.97\textwidth}[t]{l@{\extracolsep{\fill}}r}
      \textbf{#1} & #2 \\
      \textit{\small#3} & \textit{\small #4} \\
    \end{tabular*}\vspace{-7pt}
}

\newcommand{\resumeSubSubheading}[2]{
    \item
    \begin{tabular*}{0.97\textwidth}{l@{\extracolsep{\fill}}r}
      \textit{\small#1} & \textit{\small #2} \\
    \end{tabular*}\vspace{-7pt}
}

\newcommand{\resumeProjectHeading}[2]{
    \item
    \begin{tabular*}{0.97\textwidth}{l@{\extracolsep{\fill}}r}
      \small#1 & #2 \\
    \end{tabular*}\vspace{-7pt}
}

\newcommand{\resumeSubItem}[1]{\resumeItem{#1}\vspace{-4pt}}

\renewcommand\labelitemii{$\vcenter{\hbox{\tiny$\bullet$}}$}

\newcommand{\resumeSubHeadingListStart}{\begin{itemize}[leftmargin=0.15in, label={}]}
\newcommand{\resumeSubHeadingListEnd}{\end{itemize}}
\newcommand{\resumeItemListStart}{\begin{itemize}}
\newcommand{\resumeItemListEnd}{\end{itemize}\vspace{-5pt}}

%-------------------------------------------
%%%%%%  RESUME STARTS HERE  %%%%%%%%%%%%%%%%%%%%%%%%%%%%


\begin{document}

%----------HEADING----------
% \begin{tabular*}{\textwidth}{l@{\extracolsep{\fill}}r}
%   \textbf{\href{http://sourabhbajaj.com/}{\Large Sourabh Bajaj}} & Email : \href{mailto:sourabh@sourabhbajaj.com}{sourabh@sourabhbajaj.com}\\
%   \href{http://sourabhbajaj.com/}{http://www.sourabhbajaj.com} & Mobile : +1-123-456-7890 \\
% \end{tabular*}

\begin{center}
    \textbf{\Huge \scshape Alan Yuan} \\ \vspace{1pt}
    \small \href{mailto:alan.yuan.jly@gmail.com}{\underline{alan.yuan.jly@gmail.com}} $|$ 
    \href{https://linkedin.com/in/jalnyn}{\underline{linkedin.com/in/jalnyn}} $|$
    \href{https://github.com/jalnyn}{\underline{github.com/jalnyn}}
\end{center}


%-----------EDUCATION-----------
\section{Education}
  \resumeSubHeadingListStart
    \resumeSubheading
      {University of Toronto}{3.82 cGPA}
      {BSc Computer Science Specialist, Major in Mathematics}{Sep. 2018 -- May 2023}
            \resumeItem{\textbf{Relevant Coursework}: Data Structures and Algorithms (A+), Operating Systems (A+), Parallel Programming (A+), Neural Networks and Deep Learning (A+), Intro to AI (A+), Introduction to Machine Learning (A+), Algorithm Design, Analysis \& Complexity (A)}
  \resumeSubHeadingListEnd
%-----------EXPERIENCE-----------
\section{Work Experience}
  \resumeSubHeadingListStart
    \resumeSubheading
      {Amazon}{May 2022 -- Aug 2022}
      {Software Developer - Intern}{Vancouver, British Columbia}
      \resumeItemListStart
        \resumeItem{Engineered a microservice in \textbf{Java} to send notifications to \textbf{100mil+} customer of cashback on select products}
        \resumeItem{Ensured modularity by designing a plugin system for processing the customer orders for microservice}
        \resumeItem{Utilize \textbf{AWS} webservices such as \textbf{Lambda}, \textbf{SQS} and \textbf{SNS} to ensure scalability of the notification system}
      \resumeItemListEnd

    \resumeSubheading
      {Intel}{May 2021 -- May 2022}
      {Software Engineer - Intern}{Toronto, Ontario}
      \resumeItemListStart
        \resumeItem{Developed flagship product using \textbf{C\texttt{+}\texttt{+}}, \textbf{Python} and \textbf{Bash} for speedup by re-routing the compilation}
        \resumeItem{Developed support software to generate 4000+ of completely random test-cases for edge-case testing}
        \resumeItem{Optimized support tool's Ram templates to reduce false positives and failing cases by around \textbf{70\%}}
        \resumeItem{Implemented various new features and upgrades such as re-scripting tools to utilized new control system, dynamic database size notifier, hierarchy re-router for missing entities, automatic parameter setting aggregator and more}
        \resumeItem{Migrated all 600+ failing regression test cases to the new compilation flow leading to decreaesed build failures}
      \resumeItemListEnd
      
% -----------Multiple Positions Heading-----------
%    \resumeSubSubheading
%     {Software Engineer I}{Oct 2014 - Sep 2016}
%     \resumeItemListStart
%        \resumeItem{Apache Beam}
%          {Apache Beam is a unified model for defining both batch and streaming data-parallel processing pipelines}
%     \resumeItemListEnd
%    \resumeSubHeadingListEnd
%-------------------------------------------

    \resumeSubheading
      {Centivizer}{April 2020 - Sept 2020}
      {Software Developer - Part-time}{Toronto, Ontario}
      \resumeItemListStart
        \resumeItem{Designed and wrote backend application using \textbf{Node.JS} and \textbf{SimplePeer} to connect users via video call}
        \resumeItem{Establish communication between client and backend for video feed using \textbf{socket.io}}
        \resumeItem{Integrated video feature with user database through \textbf{RESTful API} using the \textbf{Axios} Library}
        \resumeItem{Decreased server load by re-working notification system to use a socket based approach}
    \resumeItemListEnd
   \resumeSubHeadingListEnd

% \section{Research Experiences}
%     \resumeSubheading
%       {PAIR Lab }{Sept 2021 - present}
%       {Lab Assistant - Part-time}{Toronto, Ontario}
%       \resumeItemListStart
%         \resumeItem{Develop randomized scenes for Reinforcement Robot Learning }
%     \resumeItemListEnd

%   \resumeSubHeadingListEnd



\section{Research Projects}
    \resumeSubHeadingListStart
        \resumeProjectHeading
          {\textbf{PAIR lab assistant} $|$ Private repo (paper under review RA-L - IEEE)}{Sep 2021 -- present}
          \resumeItemListStart
            \resumeItem{Built on top of \textbf{NVIDIA}'s Isaacsim to create a robot reinforcment learning \textbf{benchmark}}
            \resumeItem{Implementing a task-flow and enviroment randomizer for \textbf{causal reward} based research}
            \resumeItem{Creating physics scenes and testing \textbf{reinforcment learning} algorithms to be used as a benchmark}
            \resumeItem{Utilize \textbf{PyTorch} and \textbf{PPO} implementations such as \textbf{rslrl}, \textbf{rlgames} and \textbf{rllib}}
            \resumeItem{Setup and trained a variaty of robots including \textbf{frankas and mobile manipulators}}
          \resumeItemListEnd

        \resumeProjectHeading
          {\textbf{CaNetDa: Deep learning for GeoGuesser in Canada} $|$ Link: \href{https://github.com/st-tran/CSC413-Project}{GitHub}}{Jan 2021 -- April 2021}
          \resumeItemListStart
            \resumeItem{Utilized a deep learning approach utilizing multiple deep learning techniques to have an AI play GeoGuesser.}
            \resumeItem{Utilized a \textbf{PyTorch} implementation of \textbf{ResNet}, \textbf{EfficientNet} and \textbf{Vision Transformer} to predict the location}
            \resumeItem{With our approach, a accuracy of \textbf{60\%} was consistently achieved out of 13 options}
          \resumeItemListEnd
      \resumeProjectHeading
          {\textbf{Introduction to Machine Learning course competition} $|$ Link: \href{https://github.com/JaLnYn/csc311_final_project/tree/main/project/starter_code}{GitHub}}{Sept 2020 -- Dec 2020}
          % \href{https://github.com/st-tran/CSC413-Project}{GitHub}
          \resumeItemListStart
            \resumeItem{Achieved the 5th highest score in the competition and a 99\% on the project write-up.}
            \resumeItem{ Chose and implemented a Matrix Factorization SGD algorithm to recommend a selection of movies to users.}
            \resumeItem{Improved on the SGD training process by adding weight regularization and biases based on reseach papers}
            \resumeItem{Used ensembles to decrease variance ensuring the private score will be similar to that of the validation set.}
          \resumeItemListEnd
    \resumeSubHeadingListEnd

    


%-----------PROJECTS-----------
\section{Projects}
    \resumeSubHeadingListStart
      \resumeProjectHeading
        {\textbf{Deep QLearning snake} $|$ Link: \href{https://github.com/JaLnYn/pokerbot}{GitHub}}{May 2021 -- present}
        % {\textbf{Deep QLearning snake} $|$}{June 2021 -- July 2021}
        \resumeItemListStart
          \resumeItem{Utilized \textbf{PyTorch} to write a Deep Q-Learning algorithm}
          \resumeItem{Played the snake game with DQL agent reaching a high score of \textbf{40} after \textbf{5} minuites of training}
        \resumeItemListEnd
        \resumeProjectHeading
        {\textbf{CFR Minimization (Kuhn Poker, Tic-Tac-Toe and Coup)} $|$ Link: \href{https://github.com/JaLnYn/pokerbot}{GitHub}}{May 2021 -- July 2021}
        %\href{https://github.com/JaLnYn/pokerbot}{GitHub}
        \resumeItemListStart
          \resumeItem{Developing a general framework to find nash equilibrium using CFR, CFR+ and MCCFR}
          \resumeItem{Implemented each of the algorithms to play tic-tac-toe and Kuhn poker}
        \resumeItemListEnd
        \resumeProjectHeading
          {\textbf{Tenant-Landlord Matching App} $|$ Links: \href{https://github.com/JaLnYn/pk2_ser}{server-side}, \href{https://github.com/JaLnYn/pk2_cli}{client-side}}{Aug 2020 -- June 2021}
          \resumeItemListStart
            \resumeItem{Fullstack development of an mobile application to match landlords and tenants}
            \resumeItem{Constructed front-end using \textbf{React Native} and common packages such as \textbf{React Navigation} and \textbf{axios}}
            \resumeItem{Features: Authentication, Images upload utilizing \textbf{multer}, Tinder-like swiping, instant messaging with \textbf{Socket.io}}
            \resumeItem{Utilized \textbf{Node.js}, \textbf{GraphQL}, and database \textbf{Postgres} to construct backend}
          \resumeItemListEnd
      \resumeProjectHeading
        {\textbf{Tron UDP multiplayer} $|$ Link: \href{https://github.com/JaLnYn/Tron}{GitHub}}{Sep 2019 -- Dec 2019}
        \resumeItemListStart
          \resumeItem{Created a four player game for local networks using the \textbf{UDP} network protocol and C\texttt{++}}
          \resumeItem{\textbf{Forked} timer from the server to ensure the game runs on time}
          \resumeItem{Utilize \textbf{epoll} for both client and server to monitor the socket as well as the timer (server) and stdin (client)}
        \resumeItemListEnd
      \resumeProjectHeading
        {\textbf{BF-interpreter} $|$ Link: \href{https://github.com/JaLnYn/Bf-interpreter}{GitHub}}{Mar 2018 -- Nov 2018}
        \resumeItemListStart
          \resumeItem{Built \textbf{interpreter} that runs BF in C}
          \resumeItem{Reads user input in \textbf{real-time} as BF shell and reads BF files}
          \resumeItem{Runs all example BF programs found on \href{https://en.wikipedia.org/wiki/Esoteric_programming_language\#Binary_lambda_calculus}{\underline{wikipedia}}}
        \resumeItemListEnd
    \resumeSubHeadingListEnd



%
%-----------PROGRAMMING SKILLS-----------
\section{Technical Skills}
 \begin{itemize}[leftmargin=0.15in, label={}]
    \small{\item{
     \textbf{Languages}{: Python, C/C++, JavaScript, Java, C\#, R} \\
     \textbf{Tools}{: Git, React Native, Node.js, MongoDB, SQL (Postgres), PyTorch, Numpy, Pandas, GDB, GraphQL} \\
    }}
 \end{itemize}


%-------------------------------------------
\end{document}

