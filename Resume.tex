%-------------------------------------
% LaTeX Resume for Software Engineers
% Author : Leslie Cheng
% License : MIT
%-------------------------------------
% https://github.com/lcfyi/software-resume-template
\documentclass[letterpaper,12pt]{article}[leftmargin=*]
\input{glyphtounicode}

\pdfgentounicode=1

\usepackage[empty]{fullpage}
\usepackage{enumitem}
\usepackage{ifxetex}
\ifxetex
  \usepackage{fontspec}
  \usepackage[xetex]{hyperref}
\else
  \usepackage[utf8]{inputenc}
  \usepackage[T1]{fontenc}
  \usepackage[pdftex]{hyperref}
\fi
\usepackage{fontawesome}
\usepackage[sfdefault,light]{FiraSans}
\usepackage{anyfontsize}
\usepackage{xcolor}
\usepackage{tabularx}

%-------------------------------------------------- SETTINGS HERE --------------------------------------------------
% Header settings
\def \fullname {Alan Yuan}
\def \subtitle {}

\def \linkedinicon {\faLinkedin}
\def \linkedinlink {https://linkedin.com/in/jalnyn/}
\def \linkedintext {/JaLnYn}

\def \phoneicon {\faPhone}
\def \phonetext {+1-647-918-8482}

\def \emailicon {\faEnvelope}
\def \emaillink {mailto:alan.yuan.jly@gmail.com}
\def \emailtext {alan.yuan.jly@gmail.com}

\def \githubicon {\faGithub}
\def \githublink {https://github.com/JaLnYn}
\def \githubtext {/JaLnYn}

\def \websiteicon {\faGlobe}
\def \websitelink {https://google.com/}
\def \locationtext {763 Bay Street, Toronto, ON}

\def \headertype {\doublecol} % \singlecol or \doublecol

% Misc settings
\def \entryspacing {-0pt}

\def \bulletstyle {\faAngleRight}

% Define colours
\definecolor{primary}{HTML}{000000}
\definecolor{secondary}{HTML}{ff6924}
\definecolor{accent}{HTML}{263238}
\definecolor{links}{HTML}{1565C0}

%------------------------------------------------------------------------------------------------------------------- 

% Defines to make listing easier
\def \linkedin {\linkedinicon \hspace{3pt}\href{\linkedinlink}{\linkedintext}}
\def \phone {\phoneicon \hspace{3pt}{ \phonetext}}
\def \email {\emailicon \hspace{3pt}\href{\emaillink}{\emailtext}}
\def \github {\githubicon \hspace{3pt}\href{\githublink}{\githubtext}}
\def \location {\websiteicon \hspace{3pt}{\locationtext}}

% Adjust margins
\addtolength{\oddsidemargin}{-0.55in}
\addtolength{\evensidemargin}{-0.55in}
\addtolength{\textwidth}{1.1in}
\addtolength{\topmargin}{-0.6in}
\addtolength{\textheight}{1.1in}

% Define the link colours
\hypersetup{
    colorlinks=true,
    urlcolor=links,
}

% Set the margin alignment 
\raggedbottom
\raggedright
\setlength{\tabcolsep}{0in}

%-------------------------
% Custom commands

% Sections
\renewcommand{\section}[2]{\vspace{5pt}
  \colorbox{secondary}{\color{white}\raggedbottom\normalsize\textbf{{#1}{\hspace{7pt}#2}}}
}

% Entry start and end, for spacing
\newcommand{\resumeEntryStart}{\begin{itemize}[leftmargin=2.5mm]}
\newcommand{\resumeEntryEnd}{\end{itemize}\vspace{\entryspacing}}

% Itemized list for the bullet points under an entry, if necessary
\newcommand{\resumeItemListStart}{\begin{itemize}[leftmargin=4.5mm]}
\newcommand{\resumeItemListEnd}{\end{itemize}}

% Resume item
\renewcommand{\labelitemii}{\bulletstyle}
\newcommand{\resumeItem}[1]{
  \item\small{
    {#1 \vspace{-2pt}}
  }
}

% Entry with title, subheading, date(s), and location
\newcommand{\resumeEntryTSDL}[4]{
  \vspace{-1pt}\item[]
    \begin{tabularx}{0.97\textwidth}{X@{\hspace{60pt}}r}
      \textbf{\color{primary}#1} & {\firabook\color{accent}\small#2} \\
      \textit{\color{accent}\small#3} & \textit{\color{accent}\small#4} \\
    \end{tabularx}\vspace{-6pt}
}

% Entry with title and date(s)
\newcommand{\resumeEntryTD}[2]{
  \vspace{-2pt}\item[]
    \begin{tabularx}{0.97\textwidth}{X@{\hspace{60pt}}r}
      \textbf{\color{primary}#1} & {\firabook\color{accent}\small#2} \\
    \end{tabularx}\vspace{-6pt}
}

% Entry for special (skills)
\newcommand{\resumeEntryS}[2]{
  \item[]\small{
    \textbf{\color{primary}#1 }{ #2 \vspace{-6pt}}
  }
}

% Double column header
\newcommand{\doublecol}[6]{
  \begin{tabularx}{\textwidth}{Xr}
    {
      \begin{tabular}[c]{l}
        \fontsize{35}{45}\selectfont{\color{primary}{{\textbf{\fullname}}}} \\
        {\textit{\subtitle}} % You could add a subtitle here
      \end{tabular}
    } & {
      \begin{tabular}[c]{l@{\hspace{1.5em}}l}
        {\small#4} & {\small#1} \\
        {\small#5} & {\small#2} \\
        {\small#6} & {\small#3}
      \end{tabular}
    }
  \end{tabularx}
}

% Single column header
\newcommand{\singlecol}[6]{
  \begin{tabularx}{\textwidth}{Xr}
    {
      \begin{tabular}[b]{l}
        \fontsize{35}{45}\selectfont{\color{primary}{{\textbf{\fullname}}}} \\
        {\textit{\subtitle}} % You could add a subtitle here
      \end{tabular}
    } & {
      \begin{tabular}[c]{l}
        {\small#1} \\
        {\small#2} \\
        {\small#3} \\
        {\small#4} \\
        {\small#5} \\
        {\small#6}
      \end{tabular}
    }
  \end{tabularx}
}

\begin{document}
%-------------------------------------------------- BEGIN HERE --------------------------------------------------

%---------------------------------------------------- HEADER ----------------------------------------------------

\headertype{\linkedin}{\github}{}{\phone}{\email}{} % Set the order of items here
\vspace{-10pt} % Set a negative value to push the body up, and the opposite
 %-------------------------------------------------- EXPERIENCE --------------------------------------------------
\section{\faPieChart}{Experience}
 
\resumeEntryStart
  \resumeEntryTSDL
    {Software Developer | Centivizer}{ Node.JS: SimplePeer, Socket.io, Axios }{ Summer 2020}{}{}{}
  \resumeItemListStart
    \resumeItem {Designed and built server that connects "friends" to a video call with NodeJS and SimplePeer}
    \resumeItem {Used Socket.io to communicate between the client and backend.}
    \resumeItem {Connected the server with the database through a RESTful API using the Axios library.}
  \resumeItemListEnd
\resumeEntryEnd


%-------------------------------------------------- EDUCATION --------------------------------------------------
\section{\faGraduationCap}{Education} 
  \resumeEntryStart
    \resumeEntryTSDL
      {University of Toronto}{Sep, 2018 -- Jan, 2023}
      {Candidate for Honours B.S. in Computer Science}{cGPA: 3.8/4.0}
	\resumeEntryS{Relevant Coursework:} {Introduction to M.L (95), Data Structures and Algorithms (92), Operating Systems (91), Parallel Programming (94), NNs and Deep Learning (96), Intro to AI (97)}
  \resumeEntryEnd

--https://github.com/JaLnYn/pokerbot

%-------------------------------------------------- PROGRAMMING SKILLS --------------------------------------------------
\section{\faGears}{Skills}
\resumeEntryStart
 \resumeEntryS{Languages } {C/C++, Java, Python, C\#, JavaScript}
 \resumeEntryS{Tools } {PyTorch, numpy, Firebase, MongoDB, MySQL, SFML, Node.JS, React.JS, React Native}
\resumeEntryEnd

%-------------------------------------------------- PROJECTS --------------------------------------------------
\section{\faFlask}{Side Projects}
  \resumeEntryStart
    \resumeEntryTSDL
	  {Q-learning M.L algorithm (to play Snake)}{Python, PyTorch}{Dec, 2020-Jan, 2021}{\href{https://github.com/JaLnYn/}{github.com/JaLnYn/mlsnake}}
    \resumeItemListStart
      \resumeItem {Read articles and papers to understand Q-Learning.}
      \resumeItem {Implemented a Q-learning snake on top of a existing implementation of the snake-game.}
    \resumeItemListEnd
  \resumeEntryEnd 
  \resumeEntryStart
    \resumeEntryTSDL
	  {CSC311 (Intro to M.L) Course Competition}{Python, numpy}{Nov, 2020-Dec, 2021}{-}
    \resumeItemListStart
	  \resumeItem {Achieved the \textbf{5th highest score} in the competition and a 99\% on the project write-up.}
      \resumeItem {Chose and implemented a Matrix Factorization algorithm to recommend a selection of movies to users.}
      \resumeItem {Improved on the SGD training process by adding weight regularization and biases based on paper on a different application.}
      \resumeItem {Used ensembles to decrease variance ensuring the private score will be similar to that of the validation set.}
    \resumeItemListEnd
  \resumeEntryEnd 
  \resumeEntryStart
    \resumeEntryTSDL
      {Merchant Sensei Scraper}{ Python: boto3, bs4}{ Dec, 2019-present}{\href{https://merchantsensei.com/}{merchantsensei.com}}{}{}
    \resumeItemListStart
	  \resumeItem {Created \textbf{script to scrape the web} for HTMLs and other useful information to be run on EC2s.}
      \resumeItem {Using Python's threading capabilities, gave script ability to scale with CPU power}
      \resumeItem {Automated the packaging of the extracted data} 
      \resumeItem {Sends ZIP files to AWS bucket in a nice ZIP file to minimize storage costs}
    \resumeItemListEnd
  \resumeEntryEnd


  \resumeEntryStart
    \resumeEntryTSDL
      {Tron UDP multiplayer}{C++, ncurses}{Sep, 2019-Dec, 2019}{\href{https://github.com/JaLnYn/Tron}{github.com/JaLnYn/Tron}}
    \resumeItemListStart
	\resumeItem {Created a four player game for \textbf{local networks} using the UDP network protocol and C++}
    \resumeItem {Forked timer from the server to ensure the game runs on time}
    \resumeItem {Utilize epoll for both client and server to monitor the socket as well as the timer (server) and stdin (client)}
    \resumeItemListEnd
  \resumeEntryEnd

%  \resumeEntryStart
%    \resumeEntryTSDL
%      {Web-Chat | (Hack the North)}{Firebase Realtime DB, Azure M.L. API, ReactJS, JavaScript }{Sep, 2019-Sep, 2019}{\href{https://github.com/JaLnYn/DASH}{github.com/JaLnYn/DASH}}
%    \resumeItemListStart
%      \resumeItem {Lead a team of 4 to create a web-chat-app that matches strangers based on their current emotion determined by \textbf{Azure ML API}}
%      \resumeItem {Built the frontend of the web-app utilizing \textbf{Firebase} and \textbf{ReactJS}\\ }
%      \resumeItem {Utilized \textbf{Firebase Realtime Database} to easily facilitate the chat rooms}
%    \resumeItemListEnd
%  \resumeEntryEnd

  \resumeEntryStart
    \resumeEntryTSDL
      {Evolutionary M.L algorithm (to play Snake)}{C++}{Mar, 2018-Apr, 2018}{\href{https://github.com/JaLnYn/Machine-learning-Snake/tree/master/mlGame}{github.com/JaLnYn/Machine-learning-Snake}}
    \resumeItemListStart
	  \resumeItem {Read articles and papers to understand and implement the N.E.A.T Evolutionary algorithm with raw C++.}
      \resumeItem {Tested multiple fitness metrics such as score, survival time, and time between scoring. }
    \resumeItemListEnd
  \resumeEntryEnd

%   \resumeEntryStart
%    \resumeEntryTSDL
%        {Esoteric Language interpreter}{C}{Nov, 2017 - June, 2018}{\href{https://github.com/JaLnYn/Bf-interpreter}{github.com/JaLnYn/Bf-interpreter}}
%    \resumeItemListStart
%      \resumeItem {Utilized recursion and memory management to implement a simple interpreter for the Esoteric Language: BF}
%    \resumeItemListEnd
%  \resumeEntryEnd
    
  
 
\end{document}
